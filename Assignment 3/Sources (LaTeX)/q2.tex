\graphicspath{ {images/} }

\titledquestion{Analyzing NMT Systems}[25]

\begin{parts}

    \part[3] Look at the {\monofam{src.vocab}} file for some examples of phrases and words in the source language vocabulary. When encoding an input Mandarin Chinese sequence into ``pieces'' in the vocabulary, the tokenizer maps the sequence to a series of vocabulary items, each consisting of one or more characters (thanks to the {\monofam{sentencepiece}} tokenizer, we can perform this segmentation even when the original text has no white space). Given this information, how could adding a 1D Convolutional layer after the embedding layer and before passing the embeddings into the bidirectional encoder help our NMT system? \textbf{Hint:} each Mandarin Chinese character is either an entire word or a morpheme in a word. Look up the meanings of 电, 脑, and 电脑 separately for an example. The characters 电 (electricity) and  脑 (brain) when combined into the phrase 电脑 mean computer.

    {\color{red}
        \textbf{Answer:}
        By applying a 1D convolutional layer after the embedding layer, the model can learn local patterns and subword features (e.g., how the characters 电 “electricity” and 脑 “brain” combine to mean “computer”). This helps the network to capture meaningful n-gram features that improve the quality of the encoded representation before it is passed to the bidirectional encoder.
    }

    \part[8] Here we present a series of errors we found in the outputs of our NMT model (which is the same as the one you just trained). For each example of a reference (i.e., `gold') English translation, and NMT (i.e., `model') English translation, please:
    
    \begin{enumerate}
        \item Identify the error in the NMT translation.
        \item Provide possible reason(s) why the model may have made the error (either due to a specific linguistic construct or a specific model limitation).
        \item Describe one possible way we might alter the NMT system to fix the observed error. There are more than one possible fixes for an error. For example, it could be tweaking the size of the hidden layers or changing the attention mechanism.
    \end{enumerate}
    
    Below are the translations that you should analyze as described above. Only analyze the underlined error in each sentence. Rest assured that you don't need to know Mandarin to answer these questions. You just need to know English! If, however, you would like some additional color on the source sentences, feel free to use a resource like \url{https://www.archchinese.com/chinese_english_dictionary.html} to look up words. Feel free to search the training data file to have a better sense of how often certain characters occur.

    \begin{subparts}
        \subpart[2]
        \textbf{Source Sentence:} 贼人其后被警方拘捕及被判处盗窃罪名成立。 \newline
        \textbf{Reference Translation:} \textit{\underline{the culprits were} subsequently arrested and convicted.}\newline
        \textbf{NMT Translation:} \textit{\underline{the culprit was} subsequently arrested and sentenced to theft.}
        
        {\color{red}
            \textbf{Answer:}
            \begin{itemize}
                \item \textbf{Error:} The NMT output uses the singular “culprit” instead of the plural “culprits” and translates “convicted” as “sentenced to theft” (the underlined parts differ from the reference).
                \item \textbf{Possible Reason:} The model may not have learned proper number agreement or may be influenced by frequent phrase patterns in the training data that favor singular forms.
                \item \textbf{Possible Fix:} Introduce mechanisms (or additional training examples) that emphasize subject–verb and number agreement, or incorporate a loss term that penalizes semantic mismatches.
            \end{itemize}
        }

        \subpart[2]
        \textbf{Source Sentence}: 几乎已经没有地方容纳这些人,资源已经用尽。\newline
        \textbf{Reference Translation}: \textit{there is almost no space to accommodate these people, and resources have run out.   }\newline
        \textbf{NMT Translation}: \textit{the resources have been exhausted and \underline{resources have been exhausted}.}
        
        {\color{red}
            \textbf{Answer:}
            \begin{itemize}
                \item \textbf{Error:} The translation erroneously repeats “resources have been exhausted” instead of providing a full translation of both parts of the sentence.
                \item \textbf{Possible Reason:} The decoder might be suffering from exposure bias or a weakness in handling long-range dependencies, leading to repetition.
                \item \textbf{Possible Fix:} Implement a coverage mechanism or use repetition penalties during beam search to discourage redundant output.
            \end{itemize}
        }

        \subpart[2]
        \textbf{Source Sentence}: 当局已经宣布今天是国殇日。 \newline
        \textbf{Reference Translation}: \textit{authorities have announced \underline{a national mourning today.}}\newline
        \textbf{NMT Translation}: \textit{the administration has announced \underline{today's day.}}
        
        {\color{red}
            \textbf{Answer:}
            \begin{itemize}
                \item \textbf{Error:} The translation outputs “today’s day” instead of conveying the meaning of “national mourning.”
                \item \textbf{Possible Reasons:}
                \begin{itemize}
                    \item \textbf{Attention Misalignment:} The attention mechanism may focus on incorrect source tokens.
                    \item \textbf{Idiomatic Handling:} Lack of sufficient examples of idiomatic, culturally-specific expressions may lead to a literal translation.
                    \item \textbf{Tokenization Errors:} Incorrect segmentation can break down compound meanings.
                    \item \textbf{Exposure Bias:} Early decoding errors can propagate and reinforce wrong outputs.
                    \item \textbf{Overgeneralization:} Frequent patterns in the training data might bias the model toward common temporal phrases.
                \end{itemize}
                \item \textbf{Possible Fixes:} Enhance attention alignment, improve tokenization, enrich the training data with culturally diverse examples, and mitigate exposure bias.
            \end{itemize}
        }
        
        \subpart[2] 
        \textbf{Source Sentence\footnote{This is a Cantonese sentence! The data used in this assignment comes from GALE Phase 3, which is a compilation of news written in simplified Chinese from various sources scraped from the internet along with their translations. For more details, see \url{https://catalog.ldc.upenn.edu/LDC2017T02}. }:} 俗语有云:``唔做唔错"。\newline
        \textbf{Reference Translation:} \textit{\underline{`` act not, err not "}, so a saying goes.}\newline
        \textbf{NMT Translation:} \textit{as the saying goes, \underline{`` it's not wrong. "}}

        {\color{red}
            \textbf{Answer:}
            \begin{itemize}
                \item \textbf{Error:} The output “it’s not wrong” fails to capture the proverbial meaning of “唔做唔错” (which implies “if you don’t act, you won’t err”).
                \item \textbf{Possible Reason:} Idioms and proverbs are challenging because their meanings are non-compositional; the model may have opted for a literal translation instead of the intended idiomatic meaning.
                \item \textbf{Possible Fix:} Integrate an idiom database or include more idiom-rich training examples so that the model learns to handle such expressions appropriately.
            \end{itemize}
        }
        
    \end{subparts}


    \part[14] BLEU score is the most commonly used automatic evaluation metric for NMT systems. It is usually calculated across the entire test set, but here we will consider BLEU defined for a single example.\footnote{This definition of sentence-level BLEU score matches the \texttt{sentence\_bleu()} function in the \texttt{nltk} Python package. Note that the NLTK function is sensitive to capitalization. In this question, all text is lowercased, so capitalization is irrelevant. \\ \url{http://www.nltk.org/api/nltk.translate.html\#nltk.translate.bleu_score.sentence_bleu}
    } 
    Suppose we have a source sentence $\bs$, a set of $k$ reference translations $\br_1,\dots,\br_k$, and a candidate translation $\bc$. To compute the BLEU score of $\bc$, we first compute the \textit{modified $n$-gram precision} $p_n$ of $\bc$, for each of $n=1,2,3,4$, where $n$ is the $n$ in \href{https://en.wikipedia.org/wiki/N-gram}{n-gram}:
    \begin{align}
        p_n = \frac{ \displaystyle \sum_{\text{ngram} \in \bc} \min \bigg( \max_{i=1,\dots,k} \text{Count}_{\br_i}(\text{ngram}), \enspace \text{Count}_{\bc}(\text{ngram}) \bigg) }{\displaystyle \sum_{\text{ngram}\in \bc} \text{Count}_{\bc}(\text{ngram})}
    \end{align}
     Here, for each of the $n$-grams that appear in the candidate translation $\bc$, we count the maximum number of times it appears in any one reference translation, capped by the number of times it appears in $\bc$ (this is the numerator). We divide this by the number of $n$-grams in $\bc$ (denominator). \newline 

    Next, we compute the \textit{brevity penalty} BP. Let $len(c)$ be the length of $\bc$ and let $len(r)$ be the length of the reference translation that is closest to $len(c)$ (in the case of two equally-close reference translation lengths, choose $len(r)$ as the shorter one). 
    \begin{align}
        BP = 
        \begin{cases}
            1 & \text{if } len(c) \ge len(r) \\
            \exp \big( 1 - \frac{len(r)}{len(c)} \big) & \text{otherwise}
        \end{cases}
    \end{align}
    Lastly, the BLEU score for candidate $\bc$ with respect to $\br_1,\dots,\br_k$ is:
    \begin{align}
        BLEU = BP \times \exp \Big( \sum_{n=1}^4 \lambda_n \log p_n \Big)
    \end{align}
    where $\lambda_1,\lambda_2,\lambda_3,\lambda_4$ are weights that sum to 1. The $\log$ here is natural log.
    \newline
    \begin{subparts}
        \subpart[5] Please consider this example: \newline
        Source Sentence $\bs$: \textbf{需要有充足和可预测的资源。} 
        \newline
        Reference Translation $\br_1$: \textit{resources have to be sufficient and they have to be predictable}
        \newline
        Reference Translation $\br_2$: \textit{adequate and predictable resources are required}
        
        NMT Translation $\bc_1$: there is a need for adequate and predictable resources
        
        NMT Translation $\bc_2$: resources be sufficient and predictable to
        
        Please compute the BLEU scores for $\bc_1$ and $\bc_2$. Let $\lambda_i=0.5$ for $i\in\{1,2\}$ and $\lambda_i=0$ for $i\in\{3,4\}$ (\textbf{this means we ignore 3-grams and 4-grams}, i.e., don't compute $p_3$ or $p_4$). When computing BLEU scores, show your work (i.e., show your computed values for $p_1$, $p_2$, $len(c)$, $len(r)$ and $BP$). Note that the BLEU scores can be expressed between 0 and 1 or between 0 and 100. The code is using the 0 to 100 scale while in this question we are using the \textbf{0 to 1} scale. Please round your responses to 3 decimal places. 
        \newline

        {\color{red}
            \textbf{Answer:}
            Results are stored in \textbf{"bleu\_scores.ipynb"} file
        }
        \newline
        
        Which of the two NMT translations is considered the better translation according to the BLEU Score? Do you agree that it is the better translation?
        \newline

        {\color{red}
            \textbf{Answer:}
            The second translation is considered the better one according to the BLUE score, even though it is less grammatical. I disagree and, in my opinion, it illustrates BLEU’s bias toward n-gram overlap over fluency.
        }
        \newline
        
        \subpart[5] Our hard drive was corrupted and we lost Reference Translation $\br_1$. Please recompute BLEU scores for $\bc_1$ and $\bc_2$, this time with respect to $\br_2$ only. Which of the two NMT translations now receives the higher BLEU score? Do you agree that it is the better translation?
        \newline

        {\color{red}
            \textbf{Answer:}
            This time $c_1$ score was $0.408$ and $c_2$ score was $0.316$. I agree that the first translation is better in this case.
        }
        \newline
        
        \subpart[2] Due to data availability, NMT systems are often evaluated with respect to only a single reference translation. Please explain (in a few sentences) why this may be problematic. In your explanation, discuss how the BLEU score metric assesses the quality of NMT translations when there are multiple reference transitions versus a single reference translation.
        \newline

        {\color{red}
            \textbf{Answer:}
            When only one reference is available, BLEU can penalize valid paraphrases or alternative wordings that do not match the single reference exactly. This underestimates true translation quality because natural language allows many correct renderings; multiple references help cover this diversity
        }
        \newline
        
        \subpart[2] List two advantages and two disadvantages of BLEU, compared to human evaluation, as an evaluation metric for Machine Translation. 
        \newline

        {\color{red}
            \textbf{Answer:}
            \newline
            \textit{Advantages}
            \begin{itemize}
                \item Automatic and Fast: Enables large‐scale, reproducible benchmarking without human raters.
                \item Objective Score: Provides a clear numeric metric for comparing systems.
            \end{itemize}
            \textit{Disadvantages}
            \begin{itemize}
                \item Ignores Semantic Adequacy: Cannot capture meaning beyond surface n-gram overlap; penalizes synonyms or valid rephrasings.
                \item Poor Sentence-Level Correlation: Often misaligns with human judgments at the sentence level and favors overly short candidates when not carefully configured.
                \newline
            \end{itemize}
        }

    \end{subparts}


    \part[4] \emph{Beam search} is often employed to improve the quality of machine translation systems. While you were training the model, beam search results for the same example sentence at different iterations were also recorded in TensorBoard, and accessible in the \emph{TEXT} tab (Fig \ref{fig:beam-search-diagnostics-tensorboard}).

    The recorded diagnostic information includes json documents with the following fields: \texttt{example\_source} (the source sentence tokens), \texttt{example\_target} (the ground truth target sentence tokens), and \texttt{hypotheses} (10 hypotheses corresponding to the search result with beam size 10). Note that a predicted translation is often called \emph{hypothesis} in the neural machine translation jargon.

    \begin{subparts}
        \subpart[2] Did the translation quality improve over the training iterations for the model? Give three examples of translations of the example sentence at iterations 200, 3000, and the last iteration to illustrate your answer. For each iteration, pick the first beam search hypothesis as an example:
        
        
        
        {\color{red}
            \textbf{Answer:}
            Translation quality did improve.
            \newline
            \textbf{Here are the examples:}
            \begin{itemize}
                \item
                \textbf{Example:} \textit{"i was able to provide clarification on some of the matters which were raised at that meeting."}
                \item
                \textbf{Hypothesis - it. 200:} \textit{"it is also a number of the united nations of the united nations of the united nations."}
                \item
                \textbf{Hypothesis - it. 3000:} \textit{"i would like to clarify the question of a number of matters."}
                \item
                \textbf{Hypothesis - it. 18400:} \textit{"i also clarified a number of issues raised during the meeting."}
                \newline

            \end{itemize}
        }
        
        \subpart[2] How do various hypotheses resulting from beam search qualitatively compare? Give three other examples of hypotheses proposed by beam search at the last iteration to illustrate your answer.
        \newline
        
        {\color{red}
            \textbf{Answer:}
            Hypotheses are quite similar and only slightly differ grammatically and lexically.
            \newline
            \textbf{Here are the examples from the last iteration:}
            \begin{itemize}
                \item
                \textit{"i have also clarified a number of issues raised during the meeting."}
                \item
                \textit{"i have also clarified a number of matters raised during the meeting."}
                \item
                \textit{"i also clarification on a number of issues raised during the meeting."}
            \end{itemize}
            We can see though that the hypotheses slightly degrade in quality. Top 3 are grammatically correct, and in the third hypothesis even the word "matters" is used, which is closer to the target than the word "issue" in the first hypothesis. But the last example is grammatically incorrect.
        }
        
    \end{subparts}



    \begin{figure}
        \centering
        \includegraphics[width=0.7\textwidth]{images/example_translation_beam.jpg}
        \caption{Translation with beam search results for an example sentence are recorded in tensorboard for various iterations. The same data is available in the \texttt{outputs/beam\_search\_diagnostics/} folder in your working directory.}
        \label{fig:beam-search-diagnostics-tensorboard}
    \end{figure}
    

\end{parts}
